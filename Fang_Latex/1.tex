\documentclass{article}

\usepackage{fancyhdr}
\usepackage[UTF8]{ctex}
\usepackage{amsmath}
\usepackage{graphicx}
\usepackage{float}
\usepackage{xcolor}  % 引入 xcolor 宏包,可以使用不同颜色字



\pagestyle{fancy}

\fancyhead[L]{}
\fancyhead[C]{数字逻辑设计实验}
\fancyhead[R]{}

\fancyfoot[L]{}
\fancyfoot[C]{\thepage}
\fancyfoot[R]{}

\renewcommand{\headrulewidth}{0.4pt}
\renewcommand{\footrulewidth}{0.4pt}





\begin{document}

\begin{titlepage}
    \centering
    \vspace*{1.5cm}
    {\Huge\bfseries{数字逻辑设计实验报告}\par}
    \vspace{5cm}
    {\Large\bfseries{实验四:状态机实验}\par}
    \vspace{2cm}
    {\Large{2023313409 房效民}\par}
    \vspace{0.5cm}
    {\Large{完成日期:\today}\par}
    \vspace{0.5cm}
    {\Large{班级:20233119}\par}
    \vspace{0.5cm}
    {\Large{学期:秋季学期}\par}
    \vspace{0.5cm}
    {\Large{上课地点:T2615}\par}
    \vspace{0.5cm}
    {\Large{实验完成时间:4h}\par}
    \vspace{2cm}
\end{titlepage}


\newpage
\tableofcontents
\newpage


\section{\textbf{状态转换图}}
状态机状态转移图如下:\par
\begin{figure}[htbp]% 这里的参数决定图形的位置
    \centering % 使图片居中
    \includegraphics[scale=1]{状态转换图.png} % 设置图片宽度为文本宽度的80%
    \caption{状态转换图} % 图片标题
\end{figure}

\vspace{3cm}
状态解释:\par
IDLE:空闲态,发送高电平,编码0000\par
START:起始态,发送低电平,编码0001\par
DATA\_x:发送data数据的第x位,编码0010-1001\par
STOP:停止位,发送高电平,编码1010\par
valid:控制输入数字,每0.1秒变一次。\par
check:波特率控制器,每秒9600次。\par
{\color{red}图中红色字体表示下一个状态还是本身的输出条件}\par



\section{\textbf{UART数据桢仿真}}
\begin{figure}[htbp]% 这里的参数决定图形的位置
    \centering % 使图片居中
    \includegraphics[scale=0.8]{波形图.png} % 设置图片宽度为文本宽度的80%
    \caption{UART波形图} % 图片标题
\end{figure}

仿真分析:发送数据0xa5,dout通过数据桢格式依次发出0101001011,其中:\par
START位:0。\par
data数据:低位先传。\par
停止位:1\par

\section{\textbf{RTL图}}
\begin{figure}[htbp]% 这里的参数决定图形的位置
    \centering % 使图片居中
    \includegraphics[scale=0.6]{RTL.png} % 设置图片宽度为文本宽度的80%
    \caption{RTL} % 图片标题
\end{figure}


\begin{figure}[htbp]% 这里的参数决定图形的位置
    \centering % 使图片居中
    \includegraphics[scale=0.8]{状态寄存器.png} % 设置图片宽度为文本宽度的80%
    \caption{状态寄存器} % 图片标题
\end{figure}



\begin{figure}[htbp]% 这里的参数决定图形的位置
    \centering % 使图片居中
    \includegraphics[scale=0.8]{转移逻辑.png} % 设置图片宽度为文本宽度的80%
    \caption{转移逻辑} % 图片标题
\end{figure}


\begin{figure}[htbp]% 这里的参数决定图形的位置
    \centering % 使图片居中
    \includegraphics[scale=0.8]{输出.png} % 设置图片宽度为文本宽度的80%
    \caption{输出} % 图片标题
\end{figure}





\end{document}


